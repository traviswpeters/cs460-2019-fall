\documentstyle[12pt,moretext]{article}
%\setstretch{1.5}
\pagestyle{empty}
\title{CSCI 460---Operating Systems}
\author{ Lecture 1 \\
\\
\\
Textbook: Operating Systems --- Internals and Design Principles (9th edition)\\
by William Stallings}
\date{}
\begin{document}
\newenvironment{slide}[1]{\clearpage
                         ~ \hfill {\bf #1}  \hfill ~\\ \begin{itemize}
                         }{\end{itemize}}
\Large
\maketitle

%%%%%%%%%%%%%%%%%%%%%%%%%%%%%%%%%%%%%%%%%%%%%%%%%%%%%%%%%%%%%%
\begin{slide}{ 0. About CSCI~460}
\item Course homepage for now: http://www.cs.montana.edu/bhz or
\\ http://www.cs.montana.edu/bhz/classes/fall-2018/csci460/
\item Basic operating systems (roughly 67\%) and advanced OS (33\%)
\item Basic operating systems: memory management, processor management, device management and file management
\item Advanced operating system: threads, symmetric multiprocessing, multiprocessor scheduling, networking, security, public key cryptography
\item For the lecture parts, I will mainly focus on concepts and algorithms. 
\item Evaluation:\\
6 participation tests (in class, written, 5 will be counted, 10\%),\\
5 homeworks through D2L (20\%),\\
3 in-class tests (30\%),\\
3 programming assignments (24\%),\\
and  a final project (16\%) \\
%{\bf option 2:} 3 in-class tests (45\%, the lowest will be discarded and the highest will be counted twice), 3 assignments (30\%) and project (25\%) \\
% {\bf option 3:} in-class tests (30\%), assignments (20\%) and programming project (20\%) and final exam (30\%)
%\item To pass the course, you must get at least 30 out of 100 in the final exam.
\end{slide}

%%%%%%%%%%%%%%%%%%%%%%%%%%%%%%%%%%%%%%%%%%%%%%%%%%%%%%%%%%%%%%
\begin{slide}{ 1. Name some operating systems (OS) you know of}
\item --
\item --
\item --
\item --
\item --
\item --
\item --
\end{slide}

\begin{slide}{ 2. What is OS?}
\item OS is the part of computing system managing all of the hardware and software.
\item For example, it controls every file, device, section of memory, and every
nanosecond of processing time.
\item In the first part of this course, we will mainly focus on how OS works,
the related concepts as well as algorithms.
\item In the last twenty years, networks become more and more important in
operating systems. This turns distributed computing and network operating
systems into reality. We will cover some of this in the second part of the course.
\item Some contents on computer architectures and hardware will be covered, if necessary. (Some will be learnt through D2L homeworks.)
\vspace{3cm}
\item What will happen if you enter {\tt a.out}?
\end{slide}

\begin{slide}{ 3. What is OS composed of?}
\item {\bf 1. Memory Manager}, which is in charge of main memory.
\item {\bf 2. Processor Manager}, which decides how to allocate the Central Processing Unit (CPU).
\item {\bf 3. Device Manager}, which monitors every device, channel and control unit.
\item {\bf 4. File Manager}, which keeps track of every file in the system, including data files, assemblers, compilers and application programs.
\item {\bf 5. Network Manager}, which becomes an inseparable part of OS since 1990s and handles network communications and protocols, etc.
\end{slide}

\end{document}

\begin{slide}{ 4. Types of OS}
\item {\bf 1. Batch system}. Example: Systems processing punched cards, tapes, etc.
\vspace{2cm}
\item {\bf 2. Interactive system}. Example: DOS running on a PC.
\vspace{2cm}
\item {\bf 3. Real-time system}. Example: High speed aircraft, cruise missile.
\vspace{2cm}
\item {\bf 4. Hybrid system}. Example: Combination of batch and interactive system, e.g., CM-5.
\vspace{2cm}
\item {\bf 5. Embedded system}. Example: Kernel for a robot, elevators.
\end{slide}

\begin{slide}{ 5. A brief history of OS development}
\item {\bf 1. First generation (1940--1955)}, mainly used in military.
\vspace{3cm}
\item {\bf 2. Second generation (1955--1965)}, mainly used in business.
\vspace{3cm}
\item {\bf 3. Third generation (1960s--late 1970s)}.
\vspace{3cm}
\item {\bf 4. Post-3rd generation (late 1970s--early 1990s)}.
\vspace{3cm}
\item {\bf 5. Modern generation (mid-1990s--now)}.
\end{slide}

%\begin{slide}{ 6. Early Memory Management Systems}
%\item In the early days, a computer can only have one user at one time.
%To run a program, it must be entirely and contiguously loaded into memory.
%The memory management is therefore easy.
%\item {\bf Algorithm}: Load a job in a single-user system
%\end{slide}


\end{document}



